%%%%%%%%%%%%%%%%%%%%%%%%%%%%%%%%
%    Papersize and encoding    %
%%%%%%%%%%%%%%%%%%%%%%%%%%%%%%%%

% Size of margins can be changed here!
\usepackage[a4paper, total={6in, 8in}]{geometry}	%total={width, height}

% Basics: font, codec etc.
\usepackage[utf8]{inputenc}						% encoding: utf-8 (nordic letters)
\usepackage[T1]{fontenc}						% use 8-bit encoded fonts
\renewcommand{\sfdefault}{phv}					% changes the default font

%\usepackage[parfill]{parskip}      %Instead of indenting on a newline adds whitespace

%%%%%%%%%%%%%%%%%%%%%%%%%%%%%%%%
%      Tables and figures      %
%%%%%%%%%%%%%%%%%%%%%%%%%%%%%%%%

\usepackage{tabularx,booktabs,authblk}		    % various basic stuff for tables and more

% Figures and captions
\usepackage{caption}							% create captions for figures
\usepackage{subfig}								% create subfigures of a figure
%\usepackage{subcaption}					% create captions for subfigures
                                  %     currently off, due to conflicts

\usepackage{wrapfig}							% letting figures be in text

\usepackage{rotating}             % let any environment be rotated (figures sideways)
                                  %     \begin{sideways} or \begin{turn}{30}


%%%%%%%%%%%%%%%%%%%%%%%%%%%%%%%%
%      LaTeX Programming       %
%%%%%%%%%%%%%%%%%%%%%%%%%%%%%%%%

\usepackage{xparse}								% Scanning arguments
\usepackage{ifthen}								% Conditionals
\usepackage{calc}								% Calculations



%%%%%%%%%%%%%%%%%%%%%%%%%%%%%%%%
%          Hypermedia          %
%%%%%%%%%%%%%%%%%%%%%%%%%%%%%%%%

\usepackage{url, hyperref}						% \url{link} / \href{link}{replacing text}



%%%%%%%%%%%%%%%%%%%%%%%%%%%%%%%%
%         Stylization          %
%%%%%%%%%%%%%%%%%%%%%%%%%%%%%%%%

% Headers og footers
\usepackage{lastpage}                           % \lastpage command for numbers of pages
\usepackage{fancyhdr}                           % create cool headers and footers
\pagestyle{fancy}                               % who doesn't want their page to be fancy?

% Use of columns
\usepackage{multicol}

% Quotations
% "danish" or "british"
\usepackage[danish=guillemets]{csquotes}    	% two styles: "quotes" or >>guillemets<<
%\MakeAutoQuote{»}{«}                       	% decomment for easy macro
%\MakeAutoQuote*{›}{‹}                      	% decomment for even easier macros



%%%%%%%%%%%%%%%%%%%%%%%%%%%%%%%%
%             Math             %
%%%%%%%%%%%%%%%%%%%%%%%%%%%%%%%%

% various basic stuff
\usepackage{bm, mathtools, amsmath}

% Various symbol packages
%\usepackage{amssymb}
\usepackage[utopia]{mathdesign}						% full overwrite of the font system
\usepackage{stmaryrd}								% even more symbols

\let\mathcal\undefined								% Let's redefine the mathcal alphabet
\DeclareMathAlphabet{\mathcal}{OMS}{cmsy}{m}{n}

% Math shortcuts
\renewcommand{\d}{\, \mathrm{d}}                    % \d = differential d with a bit of spacing
\newcommand{\e}{\mathrm{e}}                         % \e = eulers number
\newcommand{\R}{\mathbb{R}}                         % \R = Real numbers
\newcommand{\N}{\mathbb{N}}                         % \N = Natural numbers
\DeclarePairedDelimiter\ceil{\lceil}{\rceil}        %ceil{arg}
\DeclarePairedDelimiter\floor{\lfloor}{\rfloor}     % floor{arg}

\newcounter{i}

\DeclareDocumentCommand \seq { g g g g } {			% \seq{x}{i}{j}{s}
	\setcounter{i}{0}								% x_i, x_i+s, ... x_j
	\IfNoValueF {#2} { \addtocounter{i}{#2} }
	\IfNoValueTF {#1} {x} {#1}
	_{ \arabic{i} },
	\IfNoValueTF {#4} {\addtocounter{i}{1}}  {\addtocounter{i}{#4}}
	\IfNoValueTF {#1} {x} {#1}
	_{ \arabic{i} },
	\dots
	\IfNoValueTF {#3} {} {
		, #1_{#3}
	}
}

\DeclareDocumentCommand \set { m g }{ 				% \sets{X}{C} = { X | C }
	 \lbrace
	 	#1 \IfNoValueF {#2} { \text{ | } #2 }
	 \rbrace
}


%%%%%%%%%%%%%%%%%%%%%%%%%%%%%%%%
%      Logic and proofs        %
%%%%%%%%%%%%%%%%%%%%%%%%%%%%%%%%

% Proofs
\usepackage{amsthm}								% Theorem package
\theoremstyle{definition}						% plain, definition, remark
%\swapnumbers									% If you want to have the number first

% Logic packages
\usepackage{lplfitch}						% fitch style proofs

%\usepackage{logicproof}					% alternative package, resembling the dBerLog book
%\setlength\subproofhorizspace{2em}			% Indent for subproofs. Changed for fresh variables



%%%%%%%%%%%%%%%%%%%%%%%%%%%%%%%%
%      Color and presets       %
%%%%%%%%%%%%%%%%%%%%%%%%%%%%%%%%

%\usepackage{xcolor}							% basic xcolor package
\usepackage[table,xcdraw]{xcolor}				% xcolor package with support for tables
\definecolor{lstComment}{rgb}{0.45,0.45,0.45}	% code: comments (Grey)
\definecolor{lstKey}{rgb}{0.13,0.21,1}			% code: primary keywords (Blue)
\definecolor{lstKey2}{rgb}{1,0.666667,0.13726}  % code: secondary keywords (Day[9] Orange)
\definecolor{lstString}{rgb}{0.1,0.65,0.1}		% code: strings (Green)
\definecolor{lstBase}{rgb}{0.0,0.0,0.0}			% code: base (Black)



%%%%%%%%%%%%%%%%%%%%%%%%%%%%%%%%
%            Tikz              %
%%%%%%%%%%%%%%%%%%%%%%%%%%%%%%%%

\usepackage{tikz}								% import basepackage
\usetikzlibrary{calc}							% Coordinate calcuations
\usetikzlibrary{positioning}                    % Relative positioning
\usetikzlibrary{shapes}                         % Defining nodeshapes and more (isa for E/R)

% Simple tree macro with compability to tikz
\usepackage{tikz-qtree}							% import simple tree macro
\usetikzlibrary{arrows}                         % arrows for trees

% Tikz settings for red-black trees
\tikzset{
  treenode/.style = {align=center, inner sep=0pt, text centered,
    font=\sffamily},
  arn_b/.style = {treenode, circle, white, font=\sffamily\bfseries, draw=black,
    fill=black, text width=1.5em},              % black node
  arn_r/.style = {treenode, circle, white, font=\sffamily\bfseries, draw=red,
    fill=red, text width=1.5em},              % red node
  arn_x/.style = {treenode, rectangle, draw=black, fill=black,
    minimum width=0.5em, minimum height=0.5em}  % nil node
}

% Tikz Automota for Turing Machines
\usetikzlibrary{automata}

% Tikz E/R diagram
\usetikzlibrary{er}

% Graphics and plots
\usepackage{graphicx}							% import basepackage for graphs
\usepackage{pgfplots}							% import pgfplots
\usepgfplotslibrary{fillbetween}				% add fillBetween command
\pgfplotsset{compat=1.10}						% choose version of pgfplots

% Macro for circle with symbol inside.
%\newcommand*\circled[1]{ \tikz[baseline=(C.base)]\node[draw,circle,inner sep=0.5pt](C) {#1};\!}



%%%%%%%%%%%%%%%%%%%%%%%%%%%%%%%%
%            Code              %
%%%%%%%%%%%%%%%%%%%%%%%%%%%%%%%%
\newcommand{\code}[1]{{\sf #1}}					% \code{X} writes X in a code-appropriate font




%%%%%%%%%%%%%%%%%%%%%%%%%%%%%%%%
%         lstlisting           %
%%%%%%%%%%%%%%%%%%%%%%%%%%%%%%%%

% Import lstlistings - beautiful sourcecode!
\usepackage{listings}


% Custom language definitions
% Definition of Pseudocode
\lstdefinelanguage{pseudocode}{
  keywords=[1]{
  	     by, by, downto, else, error, for, if, repeat, return, to, until, while, while
  	},								    		% list of keywords, first and last are not used
  keywords=[2]{
        and, and, or, NIL, NIL
  }
  sensitive=false,								% keywords are not case-sensitive
  morecomment=[l]{//},							% l is for line comment
  morecomment=[s]{/*}{*/},						% s is for start and end delimiter
  morestring=[b]"								% strings are enclosed in double quotes
}


% Settings for lstlistings
\lstset{language=pseudocode,					% choose language
  columns=flexible,								% let the box align to the width of the page
  breaklines=true,								% automatically break lines
  breakatwhitespace=true,						% automatically break should there only be white space.
  numbers=left,									% numbering: none, left, right
  numbersep=5pt,								% distance between linenumbers and code
  numberstyle=\color{lstComment},				% change style of numbering - currently grey.
  stepnumber=1,									% step between to line-numbers. 1 = each line is numbered
  showspaces=false,								% show spaces everywhere - adding particular underscores
  showstringspaces=false,						% underline spaces within strings only.
  showtabs=false,								% show tabs within strings adding particular underscores.
  escapeinside={*@}{@*},                		% if you want to add LaTeX within your code
  basicstyle=\ttfamily \color{lstBase},			% set basic color
  commentstyle=\color{lstComment},				% set color of comments
  keywordstyle=[1]\color{lstKey},				% set color of primary keywords
  keywordstyle=[2]\color{lstKey2},				% set color of secondary keywords
  stringstyle=\color{lstString},				% set color of strings
}

% lstlisting - Put it beautifully in the middle
\lstset{xleftmargin= .1\textwidth ,     							% leftmargin being 10% of the current width
  xrightmargin= .1\textwidth,           							% right margin also 10%
  frame=bottomline                      							% Draw a line on the bottom of the surrounding box
}

% lstlisting header
\DeclareCaptionFont{white}{\color{white}}                                       % fontstyle of caption
\DeclareCaptionFormat{listing}{\colorbox{gray}{\parbox{\linewidth}{#1#2#3}}}    % create nice grey boxes for captions
\captionsetup[lstlisting]{format=listing,labelfont=white,textfont=white}        % apply settings to listing
